\hyperref[tab:acronymsTable]{DREAM} is an easy-to-use application which intent is \cite{grw_GitHub} to design dynamic anticipatory governance models for food systems using digital public goods and community-centric approaches to strengthen data-driven policy making in the state of Telangana, India.

The state's main means of livelihood indeed heavily rely on agriculture, widely represented by smallholders farmers' activities. These ones are easily affected by complex problems such as climate change and the incoming raise of food demand due to the continuous population increase. In addition to that, the occurrence of \texttt{Covid-19} pandemic is recently causing further obstacles (such as food supply chains disruption) to the achievement of a resilient food system.

The final aim of Telengana's government is collecting and analyzing agriculture related real-time conditions in order to monitor and support smallholders' activity resilience capacity against the above mentioned problems.

\subsection{Purpose}
\label{sec:purpose}
This document provides a detailed description of the software architecture chosen for the DREAM application described in the Requirement Analysis and Specification Document. 

This includes architectural components and features, information about the deployment and the interfaces and a description of the adopted patterns. 

Moreover, a runtime view of the core functionalities of the S2B is provided, together with some interaction diagrams between the different components. 

Also some examples of user interface design are provided, in order to give a general idea of how the application will look like.

Finally, there is information about the implementation, integration and testing processes.


\subsection{Scope}
\label{sec:scope}
The system defines three main actors that will interact with the application: policy makers, farmers and agronomists.

Farmers can access the platform to see data relevant for them, such as weather forecasts and personalized suggestions, to insert data about their production and any problem they face, to request for help and suggestions by agronomists and other farmers and to create discussions with other farmers.

Agronomists can insert the area they are responsible of, they can receive and answer to request by farmers, they can see relevant data about their areas, such as weather forecasts, best/under-performing farmers and problems encountered by farmers, and they can also manage daily plans in order to visit periodically the farmers.

Finally, policy makers can visualize information and statistics about the overall process, they can see if the steering initiatives are producing significant results, they can ask well-performing farmers to write good practices and can assign incentives to them.



\subsection{Definitions, Acronyms, Abbreviations}
\label{sec:def_acr_abr}
In order to introduce some sort of coherence in an environment--- the physical world--- that is informal, we present in this section the sets of \textbf{definitions}, \textbf{acronyms} and \textbf{abbreviations} used in the following sections. These are supposed to be as more generic as possible, in order to provide more flexibility for the following phases of the Waterfall software lifecycle.

It is important to agree in advance on the specific keywords and terms that signal the presence of specific entity. 

\subsubsection{Definitions}
\label{sec:definitions}
\begin{table}[H]
    \setlength\arrayrulewidth{1pt}
    \centering
    \rowcolors{2}{white}{myblue!25}
    \begin{tabular}{|m{0.20\textwidth}|m{0.80\textwidth}|}
        \rowcolor{myblue}
        \hline
        \color{white}Concept & \color{white}Definition \\
        \hline
        \textsc{something}     &   something is ... \\
        \hline
    \end{tabular}
    
    \caption{\label{tab:def}Table of definitions}
    
\end{table}

% Synonims: system, application, service
\subsubsection{Acronyms}
\label{sec:acronyms}

\begin{table}[H]
    \setlength\arrayrulewidth{1pt}
    \centering
    \rowcolors{2}{white}{myblue!25}
    \begin{tabular}{|l|l|}
        \rowcolor{myblue}
        \hline
        \color{white}Acronym & \color{white}Expansion \\
        \hline
        \textsc{LoL}  &    Lots of Laughs \\
        \hline
        
    \end{tabular}
    
    \caption{\label{tab:acronymsTable}Table of acronyms}
    
\end{table}

% TODO Define verbs conventions of ISO/IEEE
% TODO Requirements should be ranked for importance or stability (from Hans Van Vliet book)
\subsection{Revision history}
\label{sec:history}
\begin{center}
    \setlength\arrayrulewidth{1pt}
    \rowcolors{2}{myblue!25}{white}
    \begin{longtable}{ll}
        
        \hline
        \rowcolor{myblue}\color{white}Date & \color{white}Description \\
        \hline
        28/11	&	sample\\
        \hline
        
        \rowcolor{white}\caption{\label{tab:history}History table}
        
    \end{longtable}
\end{center}
\subsection{Reference Documents}
\label{sec:ref_docs}
We present in this section the references we used to gather information about the good practices for the development of this document.
\begingroup
\renewcommand{\section}[2]{}%
%\renewcommand{\chapter}[2]{}% for other classes
\nocite{*}
\bibliographystyle{plain}
\bibliography{main}
\endgroup

%% TODO
%% UNDP github repository and its sublinks, in particular Digital public goods standard website
% Hans Van Vliet software engineering book
% ISO/IEE document
% World/machine paper
% RASD assignment document

% 1.3 sec:goals
%1.5 \ref{sec:def_acr_abr}
% 1.6 \label{sec:ref_docs}

%2.1 sec:prod_perspective
% 2.2 sec:product_functions
% 2.3 sec:actors
% 2.4 sec:domain_assumptions

% 3.1 sec:ex_interface_req
% 3.2 sec:fun_req
% 3.3 sec:per_req
% 3.4 sec:design_constraints
% 3.5 sec:sw_sys_attributes

\subsection{Document structure}
\label{sec:doc_struct}
The overall document is organized in 5 main sections. For each of them we provide a brief explanation of the contents of their subsections, except for the current one.

\subsection*{\textsc{\textcolor{myblue}{\ref{sect:introduction} Introduction}}}
blablabla

\subsection*{\textsc{\textcolor{myblue}{\ref{sect:architectural_design} Architectural design}}}
blablabla

\subsection*{\textsc{\textcolor{myblue}{\ref{sect:user_interface_design} User interface design}}}
blablabla

\subsection*{\textsc{\textcolor{myblue}{\ref{sect:req_traceability} Requirements traceability}}}
blablabla

\subsection*{\textsc{\textcolor{myblue}{\ref{sect:impl_int_test} Implementation, integration and testing plan}}}
blablabla


\subsection*{\textsc{\textcolor{myblue}{\ref{sect:effort} Effort spent}}}
blablabla

