\hyperref[tab:acronymsTable]{DREAM} is an easy-to-use application whose intent is \cite{grw_GitHub} to design dynamic anticipatory governance models for food systems using digital public goods and community-centric approaches to strengthen data-driven policy making in the state of Telangana, India.

The state's main means of livelihood indeed heavily rely on agriculture, widely represented by smallholders farmers' activities. These ones are easily affected by complex problems such as climate change and the incoming raise of food demand due to the continuous population increase. In addition to that, the occurrence of \texttt{Covid-19} pandemic is recently causing further obstacles (such as food supply chains disruption) to the achievement of a resilient food system.

The final aim of Telengana's government is collecting and analyzing agriculture related real-time conditions in order to monitor and support smallholders' activity resilience capacity against the above mentioned problems.

\subsection{Purpose}
\label{sec:purpose}
This document provides a detailed description of the software architecture chosen for the DREAM application described in the Requirement Analysis and Specification Document. 

This includes architectural components and features, information about the deployment and the interfaces and a description of the adopted patterns. 

Moreover, a runtime view of the core functionalities of the S2B is provided, together with some interaction diagrams between the different components. 

Also some examples of user interface design are provided, in order to give a general idea of how the application will look like.

Finally, there is information about the implementation, integration and testing processes.


\subsection{Scope}
\label{sec:scope}
The system defines three main actors that will interact with the application: policy makers, farmers and agronomists.

Farmers can access the platform to see data relevant for them, such as weather forecasts and personalized suggestions, to insert data about their production and any problem they face, to request for help and suggestions by agronomists and other farmers and to create discussions with other farmers.


Agronomists can insert the area they are responsible of, they can receive and answer to request by farmers, they can see relevant data about their areas, such as weather forecasts, best/under-performing farmers and problems encountered by farmers, and they can also manage daily plans in order to visit periodically the farmers.

Finally, policy makers can visualize information and statistics about the overall process, they can see if the steering initiatives are producing significant results, they can ask well-performing farmers to write good practices and can assign incentives to them.



\subsection{Definitions, Acronyms, Abbreviations}
\label{sec:def_acr_abr}
In order to introduce some sort of coherence in an environment--- the physical world--- that is informal, we present in this section the sets of \textbf{definitions}, \textbf{acronyms} and \textbf{abbreviations} used in the following sections. These are supposed to be as more generic as possible, in order to provide more flexibility for the following phases of the Waterfall software lifecycle.

It is important to agree in advance on the specific keywords and terms that signal the presence of a specific entity.

\subsubsection{Definitions}
\label{sec:definitions}
\begin{table}[H]
    \setlength\arrayrulewidth{1pt}
    \centering
    \rowcolors{2}{white}{myblue!25}
    \begin{tabular}{|m{0.20\textwidth}|m{0.80\textwidth}|}
        \rowcolor{myblue}
        \hline
        \color{white}Concept & \color{white}Definition \\
        \hline
        \textsc{Event BUS}     &    With Event BUS we refer to all the system elements used to handle the database management process, in order to guarantee data consistency across the fragmented datasets. The Event BUS is the result of the implementation of different architectural patterns, such as Database per Service, Event Sourcing, Domain Event, CQRS and Saga. \\
        \hline
        \textsc{Client-Server paradigm}     &    Distributed application structure that splits functionalities between the providers of a service, called servers, and the ones that ask for it, called clients \\
        \hline
        \textsc{Service}     &    Indipendent lightweight application that implements narrowly focused functionalities\\
        \hline
        \textsc{Microservice architecture}     &    Architectural style that functionally decomposes an application into a set of services\\
        \hline
        \textsc{Software interfaces}     &    Languages, codes and messages that programs use to communicate with each other and to the hardware\\
        \hline
        \textsc{Pattern}     &    General reusable solution for the common problems that occur in software design\\
        \hline
    \end{tabular}
    
    \caption{\label{tab:def}Table of definitions}
    
\end{table}

% Synonims: system, application, service
\subsubsection{Acronyms}
\label{sec:acronyms}

\begin{table}[H]
    \setlength\arrayrulewidth{1pt}
    \centering
    \rowcolors{2}{white}{myblue!25}
    \begin{tabular}{|l|l|}
        \rowcolor{myblue}
        \hline
        \color{white}Acronym & \color{white}Expansion \\
        \hline
        \textsc{DREAM}  &    Data-dRiven prEdictive fArMing \\
        \hline
        \textsc{S2B}     &   Software to be \\
        \hline
        \textsc{DPGS}  &    Digital Public Goods Standard\\
        \hline
        \textsc{UML}  &    Unified Modeling Language\\
        \hline
        \textsc{DBMS}  &    Data base Management System \\
        \hline
        \textsc{API}  &    Application programming interface \\
        \hline
        \textsc{GUI}  &    Graphical User Interface \\
        \hline
        \textsc{CQRS}  &    Command Query Responsibility Segregation \\
        \hline
        \textsc{REST}  &    Representational State Transfer \\
        \hline
        \textsc{CRUD}  &    Create, Read, Update, Delete \\
        \hline
        \textsc{HTTPS}  &    Hypertext Transfer Protocol Secure \\
        \hline
        \textsc{URL}  &    Uniform Resource Locator \\
        \hline
        \textsc{DB}  &    Data Base \\
        \hline
        \textsc{DDD}  &    Domain Driven Design \\
        \hline
        \textsc{JSON}  &    JavaScript Object Notation \\
        \hline
        \textsc{TDD}  &    Test Driven Development \\
        \hline
        \textsc{SSL}  &    Secure Sockets Layer \\
        \hline
        
        
    \end{tabular}
    
    \caption{\label{tab:acronymsTable}Table of acronyms}
    
\end{table}

% TODO Define verbs conventions of ISO/IEEE
% TODO Requirements should be ranked for importance or stability (from Hans Van Vliet book)
\subsection{Revision history}
\label{sec:history}

\begin{center}
    \setlength\arrayrulewidth{1pt}
    \rowcolors{2}{myblue!25}{white}
    \begin{longtable}{ll}
        
        \hline
        \rowcolor{myblue}\color{white}Date & \color{white}Description \\
        \hline
        28/12 & Architecure Structure (Microservices) \\
                & Implementation, integration and test plan \\
        \hline
        29/12 & Implementation, integration and test plan \\
                & UI diagram \\
        \hline
        30/12 & Architecture Overview \\
                & component view \\
                & Architecure Structure (Microservices) \\
                & UI diagram \\
        \hline
        31/12 & Architectural Styles and Patterns \\
                & component/deployment diagram \\
                & Sequence diagram \\
                & UI diagram \\
        \hline
        01/01 & component/deployment diagram \\
         & Component view section \\
        \hline
        02/01 & Functional requirements \\
         & Architectural Styles and Patterns \\
         & requirement treceability matrix \\
        \hline
        03/01 & Architecture Overview \\
         & Implementation, integration and test plan \\
         & Architectural Styles and Patterns \\
         & Other Design Decisions \\
         & Sequence diagram \\
         & requirements traceability matrix \\
        \hline
        04/01 & Architecure Structure (Microservices) \\
         & Introduction (Purpose, Scope) \\
         & Requirements traceability (software system attributes) \\
         & Other design decisions \\
         & Sequence diagram \\
         & Component diagram \\
        \hline
        05/01 & Integration Plan \\
         & Component Interfaces \\
         & Component view \\
         & Sequence diagram \\
        \hline
        06/01 & Deployment view \\
        \hline
        07/01 & Whole document review \\
         & Components Interfaces \\
        \hline
        \hline
        \hline
        15/02 & Requirement map refined \\
         & UI diagram refined \\
        \hline
        
        \rowcolor{white}\caption{\label{tab:history}History table}
        
    \end{longtable}
\end{center}

\subsection{Reference Documents}
\label{sec:ref_docs}
We present in this section the references we used to gather information about the good practices for the development of this document.
\begingroup
\renewcommand{\section}[2]{}%
%\renewcommand{\chapter}[2]{}% for other classes
\nocite{*}
\bibliographystyle{plain}
\bibliography{main}
\endgroup

%% TODO
%% UNDP github repository and its sublinks, in particular Digital public goods standard website
% Hans Van Vliet software engineering book
% ISO/IEE document
% World/machine paper
% RASD assignment document


\subsection{Document structure}
\label{sec:doc_struct}
The overall document is organized in 5 main sections. For each of them we provide a brief explanation of the contents of their subsections, except for the current one.

\subsection*{\textsc{\textcolor{myblue}{\ref{sect:introduction} Introduction}}}

Section \ref{sec:purpose} offers a brief description of the document content. In section \ref{sec:scope} we briefly introduce the desired functionalities that the S2B should achieve, organized for each actor involved in the application's domain. Section \ref{sec:def_acr_abr} contains the list of definitions of terms and acronyms used along the document. Section \ref{sec:history} contains the history of reviewed section of the document during time passing. Finally, section \ref{sec:ref_docs} provides the list of useful sources that have been exploited to build the document.

\subsection*{\textsc{\textcolor{myblue}{\ref{sect:architectural_design} Architectural design}}}
Section \ref{sect:architectural_design} provides a complete description of the designed software architecture. In particular section \ref{sec:overview} presents the architecture diagram from an high level perspective. Section \ref{sec:component_view} shows the structure of the components that compose the platform as well as the relationship between them through a component diagram. Section \ref{sec:deployment_view} provides a deployment diagram and describes the tiers of the S2B planned architecture. In section \ref{sec:runtime_view} we present the sequence diagrams showing the main functional requirements of the application, while section \ref{sec:component_interfaces} introduces the interfaces that handle basic required operations. Finally in sections \ref{sec:styles_patterns} and \ref{sec:other_decisionS} we respectively introduce the main patterns that are useful to architect the application and the design decisions that could lead to improvements of the planned architecture.
 
\subsection*{\textsc{\textcolor{myblue}{\ref{sect:user_interface_design} User interface design}}}
Where we propose a series of diagrams showing the application user interface and its transitions in different use cases.

\subsection*{\textsc{\textcolor{myblue}{\ref{sect:req_traceability} Requirements traceability}}}
This section provides a requirements traceability matrix that justifies the design choices of the platform. In addition, a description of the attributes the software system guarantees is presented in the same chapter.

\subsection*{\textsc{\textcolor{myblue}{\ref{sect:impl_int_test} Implementation, integration and testing plan}}}
As the title suggests this chapter is splitted in different subsections that briefly describe the implementation (section \ref{sec:implementation_plan}), integration (section \ref{sec:integration_plan}) and test (\ref{sec:test_plan}) plans of the components that make up the architecture.

\subsection*{\textsc{\textcolor{myblue}{\ref{sect:effort} Effort spent}}}
Here we provide the table of the overall efforts building the documents since last update for each team member.
