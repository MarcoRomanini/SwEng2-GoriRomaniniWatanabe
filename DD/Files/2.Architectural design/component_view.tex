% ############### 2.2) COMPONENT VIEW ####################
% sources:
% https://www.ibm.com/cloud/learn/microservices#toc-common-pat--cw3xiYi
% https://www.devteam.space/blog/microservice-architecture-examples-and-diagram/


In order to reduce the complexity of the \textbf{solution domain} (cfr. \cite{bruegge2013object}, \S 6.3.2), the required functionalities of the software are here decomposed by solution domain classes. A domain consists of multiple sub-domains: replaceable parts of the system with well-defined interfaces that encapsulates the state and behavior of their contained classes. Each sub-domain corresponds to a different part of the business. By decomposing the system into relatively independent subsystems, concurrent teams can work on individual subsystems with minimal communication overhead.

Some business functionalities require the aggregation of microservices functionalities (\ie for processing complex queries about farmer's information). As described in section \ref{sec:styles_patterns}, the system should make use of the CQRS pattern to fulfill the above mentioned requirement. Here we provide the internal structure that the interested microservice components should implement to achieve this task:


% (https://microservices.io/patterns/decomposition/decompose-by-subdomain.html)
\begin{figure}[H]
	\centering
    \includegraphics[ width=0.5\textwidth]{Images/CRUD_microservice.pdf}
	\caption{\label{fig:CRUD_microservice}CQRS-based microservice}
\end{figure}


In a CQRS-based service (ref.\cite{richardson2018microservices}, \S7.2.2), the command-side domain module handles CRUD operations and is mapped to its own database. It handles simple queries, such as non join, primary key-based queries. The command side publishes domain events whenever its data changes. These events might be published using event sourcing.

A separate query module handles simple queries. It’s much simpler than the command side because it’s not responsible for implementing the business rules. The query side uses whatever kind of database makes sense for the queries that it must support. The query side has event handlers that subscribe to domain events and update
the database.

The overall component diagram is shown below:

% \begin{figure}[H]
%     \makebox[\textwidth]{\includegraphics[width=0.99\textheight, angle=90]{Images/Component Diagram2.pdf}}
%     \caption{\label{fig:component_diagram_rotated}comp diagram}
% \end{figure}

\begin{figure}[H]
    \makebox[\textwidth]{\includegraphics[width=0.99\paperwidth]{Images/Component Diagram2.pdf}}
    \caption{\label{fig:component_diagram}Component diagram}
\end{figure}

In figure \ref{fig:component_diagram} the platform is decomposed in 7 main domains:
\begin{itemize}
    \item The \textsc{\textcolor{cyan}{Account}} domain contains all the features in order to manage the user side. In fact, it includes the log in and sign up microservices. Moreover, the component provides different interfaces for different users, in order to provide additional distinct information.
    \item The \textsc{\textcolor{pink}{Message}} domain handles all different kind of message information exchanged between users, from simple suggestion and help requests to forum and good practice document requests. These microservices are completely used by farmers, and partially used by both agronomists and policy makers.
    \item The \textsc{\textcolor{purple}{Environment}} domain contains microservices that handle all the available environment information gathered by sensors along the state of Telangana. The four microservices contained in this domain manage the information of the water irrigation system, of the soil moisture, of the weather forecasts and the wildfire ones.
    \item The \textsc{\textcolor{blue}{3rd Party}} domain provides statistics mainly based on farmers information and the integration with the environment service information. This module is composed by two microservices responsible to provide reliable farmer's performance quality information and personalized suggestion.
    \item The \textsc{\textcolor{green}{Farmer}} domain contains farmer related CQRS-based microservices providing CRUD operations for production and problem information.
    \item The \textsc{\textcolor{orange}{Agronomist}} domain contains those microservices that fulfill agronomist requirements. In particular they provide utilities for managing the area they are responsible of, the daily plan and the visits confirmation.
    \item The \textsc{\textcolor{yellow}{Policy maker}} domain is composed by a single main microservice that is responsible to provide the incentive assignment functionalities.
    \item Finally, the Farmer history microservice provides READ only operations. This component uses event handlers that subscribe to events published by Farmer and 3rd party domains' services in order to keep its database updated.
\end{itemize}


% microservice integration is not anymore based on \textit{Enterprise Service Bus} (ESB) but they rely on \textit{smart endpoints and dumb pipes} style.

% source: https://medium.com/@nathankpeck/microservice-principles-smart-endpoints-and-dumb-pipes-5691d410700f