Here you can see how to include an image in your document.


% ############### PRODUCT PERSPECTIVE ####################à
\subsection{Product perspective}
DREAM is a functional multi-user software platform which purpose is to provide functionalities described in section \ref{sect:product_functions}. The system will be composed by a series of software and hardware interfaces that interact in such a way to let users manipulate shared data. It also will exploit some graphical interface packages in order to be user friendly and easy to use.
This product is designed to run on a wide variety of machines, including operating systems Mac OS X, Windows 11,10,XP and Vista and Linux, Android and iOS. The sole requirement for the user is a web browser (Safari, Firefox, Internet Explorer, Chrome etc.) with an active internet connection.

\begin{figure}[H]
	\centering
    \includegraphics[page=1, width=\textwidth]{Images/SE2 - class diagram.pdf}
	\caption{\label{fig:uml_class_diagram}High level UML diagram}
\end{figure}

\subsubsection{User interfaces} % TODO revise te verbs
\label{sect:user_interfaces}
According to the assignment document the system will interact with 3 different user classes: policy makers, farmers and agronomists. In order to be more accessible and to fulfill the above mentioned professional figures needs the application will be supported by different devices. These users should interface to the service through electronic devices with an internet connection. Users that needs to access the service will have the possibility to connect throught:
\begin{itemize}
    \item an internet browser, addressing a specific web domain (such as \textit{www.dream.com}) that permits users to sign up/in a dedicated web application;
    \item a mobile application that can be installed on smartphones (both IOS and Android).
\end{itemize}


\subsubsection{Software interfaces}
In order to improve software flexibility and quality DREAM will use a set of external software interfaces. Rather than providing specific services names, we consider reasonable referring to them as functionalities to then be specified in design phase:
\begin{description}[font=~\normalfont\scshape]
    \item[\textbf{\textcolor{myblue}{universal logins}}] \hfill \\login APIs that also provide access by using their Facebook, Twitter, or Google profile login details are good candidates in order to quickly autenthicate the user while guaranteeing security;
    \item[\textbf{\textcolor{myblue}{big data manipulation}}] \hfill \\since a wide quantity of information needs to be recorded and accessed in a distributed system fashion, DBMS APIs are necessary for data extraction performances optimization;
    \item[\textbf{\textcolor{myblue}{third party payment processing}}] \hfill \\for guaranteeing secure payment transactions online payment APIs such as PayPal should be used;
    \item[\textbf{\textcolor{myblue}{third party data science research}}] \hfill \\% TODO
\end{description}

\subsubsection{Hardware interfaces \& constraints}
DREAM system will be composed by multiple different hardware components which can be described from two points of view:
\begin{description}[font=~\normalfont\scshape]
    \item[\textbf{\textcolor{myblue}{user perspective}}] \hfill \\Since DREAM platform is accessed by users in a fully virtual fashion, the minimum required hardware interfaces are the ones that provides internet connection, input components, a screen to visualize GUI and a web browser or an application store (like smartphones, personal computers, tablets and smart TVs).
    \item[\textbf{\textcolor{myblue}{system perspective}}] \hfill \\According to the assignment, the system should be composed by hardware devices designed to gather Telangana's environment information such as soil humidity sensor and the ones responsible for the predefined water irrigation system.
\end{description}
The user hardware interfaces also represent constrains that are required in order to permit the users to interact with the systems and manipulate shared data.

\subsection{Product functions}
\label{sect:product_functions}
In this section we present the main functionalities of the S2B, described and enriched with BPMN diagrams in order to guarantee an higher level of understanding.
\subsubsection{Sign up}
This functionality lets the User to own their account to access this application(Dream(?)).
Firstly select sign up button and fill the *information required such as email, address etc.
Then if the inserted data is accepted, an e-mail is sent to the User for asking their
verification.
Lastly if all the step above are done, the User is prompted to login page.
\subsubsection{Sharing issues to get help, suggestions}
This functionality is available for farmers and agronomists. The farmer selects request 
section and the system displays the send request button, if it is clicked, all the saved 
contacts are shown. After selecting to whom to ask, insert the question in the text form,
presses send button, then the request is sent successfully.
\subsubsection{Communication(among farmer) on forum}
This functionality is required for allowing farmers to exchange their opinion. The farmer 
access forum section which presents an eventual list that contains both previous submitted 
forums and farmer’s forum replies. By selecting forum upload button, the system displays 
*insertion form containing topic/context of the thread, title and question content. If the farmer 
inserts all the required data correctly and presses the submission button, confirmation page 
is shown. Lastly clicking confirm button the forum is generated.   
\subsubsection{Visits low performer}
This functionality is necessary to organize the Visits to make the schedule shared well 
between an agronomist and a farmer. The agronomist goes to dairy plan section and the 
application displays visualize or update button. If it is clicked, the system extracts their schedule
data which is shortly expressed by a *form with day-month-yesr, farmers to be visited.
If there is a wish to modify the plan, it is possible to modify it in a form guided by selecting 
*update button.
\subsubsection{Visualize the performance data}
This functionality is used by policymakers. The policy maker clicks the farmer's performance 
button and the system visualize the list of farmers divided by their production performance. If 
there is any interested farmer, selecting their name, application shows the information of their
*production history (more in detail | of last 10 months) 

\subsection{Actors}
In this section are defined the professional figures which, according to the assignment, the system will interact with: \textbf{policy makers}, \textbf{farmers} and \textbf{agronomists}. In order to avoid redundancy, we also introduce the concept of \textbf{user} as an abstract entity that collects the above mentioned workers' common properties. As shown in figure \ref{fig:uml_class_diagram} the latter inherit user's properties.
\subsubsection{User}
It is a person who wants to use the service. In order to access the system, it has to register to the platform (the first time) and be logged in (the following times). It also requires an Internet connection to properly use the system.
\subsubsection{Policy maker}
It is someone interested in the overall performing situation of Telangana’s farmers (e.g., Telangana’s government). It is able to surf on DREAM’s website. It uses the service to visualise information about well and under-performing farmers and to understand if the steering initiatives are producing significant results.
\subsubsection{Farmer}
It is a farmer of Telangana. It is able to surf on DREAM’s website (or to use the smartphone application). It uses the service in order to discuss with other farmers, to send requests for help to an agronomist, to insert information about what / how much he is producing, to know when he will receive a visit from an agronomist. It takes advantage in using the system because it receives incentives if he is well-performing or receives help if he is under-performing.
\subsubsection{Agronomist}
It is an agronomist of Telangana. It is able to surf on DREAM’s website (or to use the smartphone application). It uses the service to answer farmers' requests for help, to visualise data concerning their area (weather forecasts, well and under-performing farmers), to visualise and update a daily plan to visit farms in the area, to confirm the execution or specify deviation from the daily plan.


\subsection{Assumptions, dependencies and constraints}

Here is the command to refer to another element (section, figure, table, ...) in the document: \emph{As discussed in Section~\ref{sect:overview} and as shown in Figure~\ref{fig:forum_generation_seq_diag}, ...}. Here is how to introduce a bibliographic citation~\cite{UNDP_GitHub}. Bibliographic references should be included in a \texttt{.bib} file. 

Table generation is a bit complicated in Latex. You will soon become proficient, but to start you can rely on tools or external services. See for instance this \href{https://www.tablesgenerator.com}{https://www.tablesgenerator.com}. 
