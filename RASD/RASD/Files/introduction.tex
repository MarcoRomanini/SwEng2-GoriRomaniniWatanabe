% This document has been prepared to help you approaching Latex as a formatting tool for your Travlendar+ deliverables. This document suggests you a possible style and format for your deliverables and contains information about basic formatting commands in Latex. A good guide to Latex is available here \href{https://tobi.oetiker.ch/lshort/lshort.pdf}{https://tobi.oetiker.ch/lshort/lshort.pdf}, but you can find many other good references on the web. 

% Writing in Latex means writing textual files having a \texttt{.tex} extension and exploiting the Latex markup commands for formatting purposes. Your files then need to be compiled using the Latex compiler. Similarly to programming languages, you can find many editors that help you writing and compiling your latex code. Here \href{https://beebom.com/best-latex-editors/}{https://beebom.com/best-latex-editors/} you have a short oviewview of some of them. Feel free to choose the one you like.  

% Include a subsection for each of the following items\footnote{By the way, what follows is the structure of an itemized list in Latex.}:
% \begin{itemize}
% \item
% Purpose: here we include the goals of the project
% \item
% Scope: here we include an analysis of the world and of the shared phenomena
% \item
% Definitions, Acronyms, Abbreviations
% \item
% Revision history
% \item
% Reference Documents 
% \item
% Document Structure
% \end{itemize}
% Below you see how to define the header for a subsection.
\hyperref[tab:acronymsTable]{DREAM} is an easy-to-use application which intent is \cite{grw_GitHub} to design dynamic anticipatory governance models for food systems using digital public goods and community-centric approaches to strengthen data-driven policy making in the state of Telangana, India.

The state's main means of livelihood indeed heavily rely on agriculture, widely represented by smallholders farmers' activities. These ones are easily affected by complex problems such as climate change and the incoming raise of food demand due to the continuous population increase. In addition to that, the occurrence of \texttt{Covid-19} pandemic is recently causing further obstacles (such as food supply chains disruption) to the achievement of a resilient food system.

The final aim of Telengana's government is collecting and analyzing agriculture related real-time conditions in order to monitor and support smallholders' activity resilience capacity against the above mentioned problems.

\subsection{Purpose}
\label{sec:purpose}
The main purpose of the \hyperref[tab:acronymsTable]{S2B} is acting as a bridge between the main actors of the food systems: \textbf{policy makers}, \textbf{farmers} and \textbf{agronomists}. This can be achieved through the development of a multi-user unified platform that allows actors to share data. In particular, the system should guarantee the following needs and functionalities based on the user identity; we present them in the following list, extracted from the original assignment.

\subsection*{\textsc{\textcolor{myblue}{Policy makers needs}}}
\label{sec:policyMakersNeeds}
    \begin{enumerate}
        \item \label{itm:first} Identify those farmers who are performing well, especially when they demonstrate to be resilient to meteorological adverse events, as these farmers will receive special incentives and will be asked to provide useful best practices to the others
        \item \label{itm:badly_performing_farmers} Identify those farmers who need to be helped as they are performing particularly badly
        \item Understand whether the steering initiatives carried out by agronomists with the help of good farmers produce significant results
    \end{enumerate}
\subsection*{\textsc{\textcolor{myblue}{Farmers needs}}}
\label{sec:farmersNeeds}
    \begin{enumerate}[resume]
        \item \label{itm:data_relevant_to_farmers} Visualize data relevant to them –-- for instance, weather forecasts, personalized suggestions concerning specific crops to plant or specific fertilizers to use --– based on their location and type of production
        \item Insert in the system data about their production and any problem they face
        \item Request for help and suggestions by agronomists and other farmers
        \item Create discussion forums with the other farmers
    \end{enumerate}
\subsection*{\textsc{\textcolor{myblue}{Agronomist needs}}}
\label{sec:agronomistsNeeds}
    \begin{enumerate}[resume]
        \item Insert the area they are responsible of;
        \item Receive information about requests for help and answer to these requests
        \item Visualize data concerning weather forecasts in the area and the best performing farmers in the area
        \item \label{itm:daily_plan} Visualize and update a daily plan to visit farms in the area, assuming that all farms must be visited at least twice a year, but those that are under-performing should be visited more often, depending on the type of problem they are facing
        \item Confirm the execution of the daily plan at the end of each day or specify the deviations from the plan
    \end{enumerate}

Referring to point \ref{itm:first}, we assume that is policy maker responsibility, given information about well performing farmers by the system, to send incentives through the platform itself. 
\newline

% point 3 of the review notes
Referring to points \ref{itm:first} and \ref{itm:badly_performing_farmers}, we assume that the system is able to provide a well desired evaluation of farmers performance status. The evaluation could be made based on their production rates, climate change resiliency and the location of their activity.
\newline

% point 2 of the review notes
Referring to point \ref{itm:data_relevant_to_farmers}, we provide an informal definition of the information that is relevant for farmers in section \ref{sec:definitions}. We also assume that weather forecasts and personalized suggestion are internal behaviors of the \hyperref[tab:acronymsTable]{S2B}.
\newline

% point 1 of the review notes
Referring to point \ref{itm:daily_plan}, we assume that the system automatically prepares for the agronomists some sketches of the daily plans. Agronomists are then allowed to edit them as they feel more comfortable.
The suggestion could be made based on their current assigned farmers information (such as the performance status and the number of previous visits) and the areas they chosen to be responsible of.
\newline

Furthermore the system should make use of the already collected data such as:
\begin{itemize}
    \item Data concerning meteorological short-term and long-term forecasts;
    \item Information obtained by the water irrigation system concerning the amount of water used by each farmer;
    \item Information obtained by sensors deployed on the territory and measuring the humidity of soil.
\end{itemize}


\subsection{Scope}
\label{sec:scope}
The environment in which the platform will be involved will be the whole state of Telangana. Policy makers, farmers and agronomist will have the possibility to access the system through their personal devices. Furthermore, the devices responsible for gathering information, such as the ones of the water irrigation system and the humidity sensors, are supposed to be shared along the territory in a distributed system fashion.

\newpage

\subsubsection{World Phenomena}
%what you write here is a comment that is not shown in the final text
In this section the environment related phenomena that the S2B is supposed to face are introduced. As described in \cite{jackson_twatm}, for each phenomenon we specify if it is shared and who controls it.

\begin{table}[H]
    \setlength\arrayrulewidth{1pt}
    \centering
    \rowcolors{2}{white}{myblue!25}
    \begin{tabular}{|l|c|c|}
        \rowcolor{myblue}
        \hline
        \color{white}Phenomenon & \color{white}Who controls it?   & \color{white}Is it shared? \\
        \hline
        % USER RELATED PHENOMENA
        User registration                                               &   M   &   Y \\
        \hline
        User login                                                      &   W   &   Y \\
        \hline
        Check usernane and password                                     &   M   &   N \\
        \hline
        Visualize the data                                              &   M   &   Y \\
        \hline
        Gather and send the information about humidity                  &   M   &   Y \\
        \hline
        Send the data automatically                                     &   M   &   Y \\
        \hline
        % POLICY MAKER RELATED PHENOMENA
        Setup the systems and locate them                               &   W   &   Y \\
        \hline
        Fetch info from the located systems                             &   M   &   Y \\
        \hline
        Gather and send the information about each crops condition      &   M   &   Y \\
        \hline
        Put the flag according to the given behaviour of each farmers   &   W   &   Y \\
        \hline
        Visualize the sorted tendency of farmers characteristics        &   W   &   Y \\
        \hline
        Adjust the threshold(distinguish good/problematic farmer)       &   W   &   Y \\
        % FARMERS RELATED PHENOMENA
        \hline
        User insert data about their production                         &   W   &   Y \\
        \hline
        User insert problems faced information                          &   W   &   Y \\
        \hline
        User asks for suggestion to agronomists and/or farmers          &   W   &   Y \\
        \hline
        System notifies addressed users about a farmer request          &   M   &   Y \\
        \hline
        Farmers problems take place                                     &   W   &   N \\
        \hline
        User create discussion forums                                   &   W   &   Y \\
        % AGRONOMISTS RELATED PHENOMENA
        \hline
        Users insert the area they are responsible for                  &   W   &   Y \\
        \hline
        User answers requests for help                                  &   M   &   Y \\
        \hline
        Visualize data concerning weather forecast in the area          &   M   &   Y \\
        \hline
        Visualize best/under performing farmers in the area             &   M   &   Y \\
        \hline
        Visualize daily plan                                            &   M   &   Y \\
        \hline
        Update daily plan                                               &   W   &   Y \\
        \hline
        Confirm the execution of the daily plan                         &   W   &   Y \\
        \hline
        Specify deviation from the plan                                 &   W   &   Y \\
        \hline
        Check farmers that are under-performing                         &   M   &   N \\
        \hline
        Check farmers that are performing well                          &   M   &   N \\
        \hline
        Check farms that need to be visited                             &   M   &   N \\
        \hline
    \end{tabular}
    
    \caption{\label{tab:phenomena_table}Phenomena table}
    
\end{table}

\newpage

\subsection{Goals}
\label{sec:goals}
According to \cite{jackson_dsfr} (revised), it is stakeholders' prerogative to determine the goals. Thus, in this section, in order to introduce some sort of atomicity of the goals, we derive a formal description of them from the requested functionalities presented in section \ref{sec:scope}. Furthermore, in order to justify the definition of each goal, we also map each of them to the related assignment extract.

\begin{table}[H]
    \setlength\arrayrulewidth{1pt}
    \centering
    \rowcolors{2}{white}{myblue!25}
    \begin{tabular}{|l|m{0.85\textwidth}|l|}
        \rowcolor{myblue}
        \hline
        \color{white}GX & \color{white}Definition   & \color{white}Cfr.\\
        \hline
        % POLICY MAKER GOALS
        G.1                                               &     Policy maker should be able to see best performing and under-performing farmers in all the areas &   \hyperref[sec:policyMakersNeeds]{1,2} \\
        \hline
        G.2                                                      &  Policy maker should be able to handle incentives for high performing farmer   &    \hyperref[sec:policyMakersNeeds]{1}\\
        \hline
        G.3                                     &   Policy maker should be able to ask high performing farmers to write good practices   &   \hyperref[sec:policyMakersNeeds]{2,3} \\
        \hline
        G.4                                              &   Policy maker should be able to compare the performance difference of farmer between the current data and the past data(before agronomist visits)     &   \hyperref[sec:policyMakersNeeds]{1} \\
        \hline
        % FARMER GOALS
        G.5                                  &   Farmers should be able to visualize data relevant to them, based on their location and type of product   &   \hyperref[sec:farmersNeeds]{4} \\
        \hline
        G.6                                                      &  Farmers should be able to manipulate data about their production and any problem faced   &    \hyperref[sec:farmersNeeds]{5}\\% should it be splitted in 2 parts?
        \hline
        G.7                                               &     Farmers should be able to request for help and suggestions by agronomists and other farmers &   \hyperref[sec:farmersNeeds]{6} \\
        \hline
        G.8                                              &   Farmers should be able to create discussion forums with other farmers     &   \hyperref[sec:farmersNeeds]{7} \\
        \hline
        % AGRONOMIST GOALS
        G.9                                  &   Agronomists should be able to insert the area they are responsible for   &   \hyperref[sec:agronomistsNeeds]{8} \\
        \hline
        G.10                                                      &  Agronomists should be able to receive information about requests for help and answer to these requests   &    \hyperref[sec:agronomistsNeeds]{9}\\% should it be splitted in 2 parts?
        \hline
        G.11                                               &     Agronomists should be able to see best performing and under-performing farmers in the area &   \hyperref[sec:agronomistsNeeds]{10} \\
        \hline
        G.12                                              &   Agronomists should be able to see data concerning weather forecasts in the area     &   \hyperref[sec:agronomistsNeeds]{10} \\
        \hline
        G.13                                               &     Agronomists should be able to visualize and update a daily plan to visit farms in the area &   \hyperref[sec:agronomistsNeeds]{11} \\
        \hline
        G.14                                              &   Agronomists should be able to confirm the execution of the daily plan at the end of each day or specify the deviations from the plan     &   \hyperref[sec:agronomistsNeeds]{12} \\
        \hline
    \end{tabular}
    
    \caption{\label{tab:goals}Goals table}
\end{table}


For each of them we provide the \textbf{traceability matrix} in section \ref{sec:traceability_matrix}.

\subsection{Definitions, Acronyms, Abbreviations}
\label{sec:def_acr_abr}
In order to introduce some sort of coherence in an environment--- the physical world--- that is informal, we present in this section the sets of \textbf{definitions}, \textbf{acronyms} and \textbf{abbreviations} used in the following sections. These are supposed to be as more generic as possible, in order to provide more flexibility for the following phases of the Waterfall software lifecycle.

It is important to agree in advance on the specific keywords and terms that signal the presence of specific entity. Following the convention defined in \cite{iso_ieee_standard}, we therefore adopt the following terms to identify and describe requirements, goals and domain assumptions.

%# SOURCE : https://reqexperts.com/2012/10/09/using-the-correct-terms-shall-will-should/
\begin{itemize}
    \item \textbf{Shall}: used to indicate a requirement that is contractually binding, meaning it must be implemented, and its implementation verified;
    \item \textbf{Will}: used to indicate a domain assumption.  Will statements are not subject to verification;
    \item \textbf{Should}: used to indicate a goal which must be addressed by the design team but is not formally verified.
\end{itemize}

\subsubsection{Definitions}
\label{sec:definitions}
\begin{table}[H]
    \setlength\arrayrulewidth{1pt}
    \centering
    \rowcolors{2}{white}{myblue!25}
    \begin{tabular}{|m{0.20\textwidth}|m{0.80\textwidth}|}
        \rowcolor{myblue}
        \hline
        \color{white}Concept & \color{white}Definition \\
        \hline
        \textsc{Production Data}     &   All general kind of information that describes what the farmer produces with his activity. We allow the farmer to describe it with a set of mandatory properties that are analogous to each product (e.g. type of product, amount of produced items, unit of measurement etc.). In addition, the user can fill some optional fields like a qualitative description of the items produced or notes relevant for other users of the system \\
        \hline
        \textsc{Problem information}  &   Every kind of information that describes in a functional way the general set of events that could get in the way of farmers' ordinary production rate \\
        \hline
        \textsc{Help request}     &   The farmer's tool to contact agronomists. The farmer should be able to ask questions in a private way to other users of the systems (farmers and agronomists) in order to increase the quality of his activity \\
        \hline
        \textsc{\nohyphens{Good  practices document}}     &   Document in which the practices that good farmers follow are described. This document is made to be submitted into the system and can be useful for those farmers who want to improve their production \\
        \hline
        \textsc{Performance}     &   With the term \textbf{performance} we refer to the evaluation of farmers, based on their productivity and resilience. Depending on their deviance (positive or negative), farmers can be classified as high-performing, normal-performing or under-performing \\
        \hline
        \textsc{Incentive}  &   With the term \textbf{incentive} we refer to some sort of voucher which can be given to farmers if they satisfy certain conditions \\
        \hline
        \textsc{Farmer's relevant information}  &   The information that is considered to concern farmers: weather forecasts, personalized suggestion, soil humidity data, water irrigation system details, visits calendar of assigned agronomists, history of past faced problems and uploaded product information \\
        \hline
    \end{tabular}
    
    \caption{\label{tab:def}Table of definitions}
    
\end{table}

\newpage

% Synonims: system, application, service
\subsubsection{Acronyms}
\begin{table}[H]
    \setlength\arrayrulewidth{1pt}
    \centering
    \rowcolors{2}{white}{myblue!25}
    \begin{tabular}{|l|l|}
        \rowcolor{myblue}
        \hline
        \color{white}Acronym & \color{white}Expansion \\
        \hline
        \textsc{DREAM}  &    Data-dRiven prEdictive fArMing \\
        \hline
        \textsc{S2B}     &   Software to be \\
        \hline
        \textsc{GDPR}  &    General Data Protection Regulation\\
        \hline
        \textsc{ECOWAS}  &    Economic Community of West African States\\
        \hline
        \textsc{DPGS}  &    Digital Public Goods Standard\\
        \hline
        \textsc{STQC}  &    Standardization Testing and Quality Certification\\
        \hline
        \textsc{UML}  &    Unified Modeling Language\\
        \hline
        \textsc{BPMN}  &    Business Process Model and Notation\\
        \hline
        \textsc{SDG}  &    Sustainable Development Goals\\
        \hline
        \textsc{MTTR}  &    Mean Time To Repair\\
        \hline
        \textsc{RBAC}  &    Role Based Access Control \\
        \hline
        \textsc{DBMS}  &    Data base Management System \\
        \hline
        \textsc{API}  &    Application programming interface \\
        \hline
        \textsc{GUI}  &    Graphical User Interface \\
        \hline
    \end{tabular}
    
    \caption{\label{tab:acronymsTable}Table of acronyms}
    
\end{table}

% TODO Define verbs conventions of ISO/IEEE
% TODO Requirements should be ranked for importance or stability (from Hans Van Vliet book)
\subsection{Revision history}
\label{sec:history}
\begin{center}
    \setlength\arrayrulewidth{1pt}
    \rowcolors{2}{myblue!25}{white}
    \begin{longtable}{ll}
        
        \hline
        \rowcolor{myblue}\color{white}Date & \color{white}Description \\
        \hline
        29/10	&	World and Machine phenomena\\
        \hline
        31/10	&	UML	class diagram\\
        \hline
        10/11	&	Goals, Domain assumptions, Requirements	\\
        \hline
        17/11	&	UML	class diagram\\
        \hline
        19/11	&	Functional requirements	\\
        \hline
        22/11	&	Actors\\
        	&	Perspective (interfaces)\\
        	&	Interface requirements\\
        \hline
        24/11	&	Perspective (interfaces)		\\
        	&	Software system attributes	\\
        	&	Interface requirements	\\
        	&	Functional requirements	\\
        \hline
        26/11	&	Performance requirements	\\
        	&	Design constraints	\\
        	&	Functional requirements\\
        	&	Product functions	\\
        	&	UI description	\\
        \hline
        29/11	&	Functional requirements	\\
        	&	Farmer BPMN\\
        \hline
        1/12	&	Alloy code\\
        	&	farmer sequence diagram	\\
        	&	Users section introduction	\\
        \hline
        3/12	&	Goals, Domain assumptions, Requirements	\\
        	&	Alloy code \\
        	&	Farmers goals mapping\\
        \hline
        06/12	&	Alloy code	\\
        	&	UML	\\
        	&	UI description	\\
        	&	Goals, Domain assumptions, Requirements		\\
        	&	Introduction	\\
        \hline
        07/12	&	Functional Requirements (sequence diagrams, scenarios)\\
        	&	UI design\\
        	&	Sequence diagram	\\
        	&	bibliography\\
        	&	latex edits, acronyms table	\\
        \hline
        10/12	&	UML diagram	\\
        	&	Functional Requirements \\
        	&	Product functions\\
        	&	Farmers scenarios	\\
        	&	Use case diagram	\\
        	&	Policy maker scenarios	\\
        \hline
        15/12	&	BPMN\\
        	&	Alloy code	\\
        	&	UML diagram		\\
        	&	Goals, Domain assumptions, Requirements		\\
        	&	Use case	\\
        	&	Transcribed goals and requirements\\
     	    &	Transcribed domain assumptions and traceability matrix  \\
        \hline
        17/12	&	UML description	\\
        	&	Added Alloy code with assertion		\\
        \hline
        20/12	&	Functional requirements	\\
        	&	Alloy code	\\
        	&	UML diagram		\\
        	&	Goals, Domain assumptions, Requirements\\
        	&	Sequence diagram	\\
        	&	Use cases	\\
        	&	Alloy code transcription	\\
        	&	Introduction revise		\\
        \hline
        21/12 & Effort table \\
         & Whole document review \\
        \hline
        
        \rowcolor{white}\caption{\label{tab:history}History table}
        
    \end{longtable}
\end{center}
\subsection{Reference Documents}
\label{sec:ref_docs}
We present in this section the references we used to gather information about the good practices for the development of this document.
\begingroup
\renewcommand{\section}[2]{}%
%\renewcommand{\chapter}[2]{}% for other classes
\nocite{*}
\bibliographystyle{plain}
\bibliography{main}
\endgroup

%% TODO
%% UNDP github repository and its sublinks, in particular Digital public goods standard website
% Hans Van Vliet software engineering book
% ISO/IEE document
% World/machine paper
% RASD assignment document

% 1.3 sec:goals
%1.5 \ref{sec:def_acr_abr}
% 1.6 \label{sec:ref_docs}

%2.1 sec:prod_perspective
% 2.2 sec:product_functions
% 2.3 sec:actors
% 2.4 sec:domain_assumptions

% 3.1 sec:ex_interface_req
% 3.2 sec:fun_req
% 3.3 sec:per_req
% 3.4 sec:design_constraints
% 3.5 sec:sw_sys_attributes

\subsection{Document structure}
\label{sec:doc_struct}
The overall document is organized in 5 main sections. For each of them we provide a brief explanation of the contents of their subsections, except for the current one.

\subsection*{\textsc{\textcolor{myblue}{\ref{sect:introduction} Introduction}}}
Here in section \ref{sec:purpose} we offer a brief description of the problem and required functionalities, with some of our assumptions about the terms used in the original assignment. In section \ref{sec:scope} we briefly describe the environment in which the \hyperref[tab:def]{S2B} will be involved and present the phenomena analysis table.
In section \ref{sec:goals} a table of the required goals is presented and defined, while section \ref{sec:def_acr_abr} contains the list of definitions of terms and acronyms used along the document. Section \ref{sec:history} contains the history of reviewed section of the document during time passing. Finally, section \ref{sec:ref_docs} provides the list of useful sources that have been exploited to build the document.

\subsection*{\textsc{\textcolor{myblue}{ Overall description}}}
This section offers a summary description about the overall organization of the system, the Hardware and Software constraints and the interfaces needed to get it work. In particular section \ref{sec:prod_perspective} proposes the UML high level class diagram structure of the application and a brief discussion about the entities contained in it. In section \ref{sec:product_functions} we provide a description of the main features offered by the application and the relatives BPMN diagrams for deeper understanding. The main actors that are supposed to interact with the application are defined in section \ref{sec:actors}, while in section \ref{sec:domain_assumptions} we define the list of domain assumptions.
\subsection*{\textsc{\textcolor{myblue}{ Specific Requirements}}}
In section \ref{sec:ex_interface_req} some mock-ups of the web and application \hyperref[tab:acronymsTable]{GUIs} are showed and described. Section \ref{sec:fun_req} represents the main portion of the document and contains the list of functional requirements described by means of actor's scenarios, use case diagrams, tables and sequence diagrams. Performance requirements are defined in section \ref{sec:per_req}, while in section \ref{sec:design_constraints} we present the design constraints. Finally in section \ref{sec:sw_sys_attributes} we provide the main expected qualitative properties the platform is required to hold.
\subsection*{\textsc{\textcolor{myblue}{ Formal Analysis using Alloy}}}
Finally, in section \ref{sec:alloy_code} we provide the Alloy code used for the analysis of the system, together with some diagrams to better understand the behaviour of the system.

\subsection*{\textsc{\textcolor{myblue}{ Effort spent}}}
Here we provide the table of the overall efforts building the documents since last update for each team member.
